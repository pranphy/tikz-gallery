%%% This diagram was made for my homework of Mathematical 
%%% Physics at Drexel University during my Masters education
%%% The homework assignment can be found here at
%%% http://physics.drexel.edu/~pgautam/courses/
\begin{tikzpicture}

    [
    %decoration={
    %  markings,
    %  mark=at position 1cm with {\arrow[line width=1pt]{>}},
    %  mark=at position .3 with {\arrow[line width=1pt]{>}},
    %  mark=at position .6 with {\arrow[line width=1pt]{>}},
    %  mark=at position 0.8 with {\arrow[line width=1pt]{>}},
    %  mark=at position -5mm with {\arrow[line width=1pt]{>}},
    %},
     on each segment/.style={
        decorate,
        decoration={
          show path construction,
          moveto code={},
          lineto code={
            \path [#1] (\tikzinputsegmentfirst) -- (\tikzinputsegmentlast);
          },
          curveto code={
            \path [#1] (\tikzinputsegmentfirst) .. controls
            (\tikzinputsegmentsupporta) and (\tikzinputsegmentsupportb) ..
            (\tikzinputsegmentlast);
          },
          closepath code={
            \path [#1]
            (\tikzinputsegmentfirst) -- (\tikzinputsegmentlast);
          },
        },
      },
    mid arrow/.style={
        postaction={decorate,
          decoration={
            markings, mark=at position .5 with {\arrow[#1]{stealth}}
          }
        }
      },
    contourline/.style={line width=1.0pt},
    axisline/.style={->,line width=0.3pt},
    ]
    \tikzmath{\R=3;\r=0.5;\X=1.1*\R;\Y=1.1*\R;}
    \draw [axisline] (-\X,0) -- (\X,0)  node [below right] {Re($z$)};
    \draw [axisline] (0,1.2*\r) -- (0,-\Y)  node[left] {Im($z$)};
    \node at (0,0) {$\times$};
    %\draw [contourline, postaction=decorate]
    \draw [contourline, postaction=={on each segment={mid arrow=red}}]
        (\r,0) node [below, font=\scriptsize] {$\epsilon$} -- 
        (\R,0) node [above] {$R$} 
        arc (0:-180:\R) node [above] {$-R$} -- 
        (-\r,0) node [below, font=\scriptsize] {$-\epsilon$} 
        arc (180:0:\r);
    \node at (\r,1.1*\r) {$\Gamma_{\varepsilon}$};
    \node at (1.15*\R*0.7,-1.15*\R*0.7) {$\Gamma_{R}$}; % 0.7 = sin(45) = cos(45)
\end{tikzpicture}

